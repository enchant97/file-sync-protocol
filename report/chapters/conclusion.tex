Testing and comparing the prototype to selected existing solutions (FTP, SMB and rsync) has shown that the final prototype can achieve a greater throughput on a network by utilising modern methodologies and building on top of UDP as the transport layer. However after minimising the overhead from prototypes one and two; prototype three was still unable to lead in reducing the overhead to be less than rsync. This is likely due to rsync having a simple connection handshake and no extra negotiation for each file.

To improve on this prototype, utilising how rsync handles small files; by placing many in a single packet could greatly increase the transfer speed and minimise on the amount of packets needed to be sent per file. Limiting latency as less packets need to traverse a network, which depending on the network or even networks could cause the transfer speed to increase.

Using UDP for this investigation, while has resulted in seeing drastic transfer speed improvements may not be feasible without the help of professionals or more future experience due to the complexities involved in handling error checking, quality of service (QoS) and congestion control.

It may however be feasible in the future to experiment with using QUIC as the transport protocol, which uses UDP and provides all of these features. Or even using SCTP in the future when it is standardised and is able to work across the internet, which is different from the existing base TCP/IP stack and allows for a much more modern approach to networking featuring the in-built ability to combine multiple packet payloads into a single packet, potentially removing the need to implement multiple messages in one packet at the application level.

This investigation did not also consider the security aspect. Which could effect transfer speed performance or what effect it may have over the overhead. Other newer protocols build on top of the existing ones investigated in this report such as TFTP, SFTP, rsync over SSH and SMB3's native encryption and message signing. At this time security of data sent over the network is greatly needed as internet traffic can be intercepted and possibly manipulated or stolen; which would not be ideal for a business if the built prototype was used in the real world.

Some of these protocols such as FTP and rsync, can also implement streaming compression allowing for file data to be compressed before transmitted potentially decreasing what is sent "over the wire". Implementing a modern compression technique could assist this prototype's overhead, however compression can effect the transfer speed due to the processing overhead.

Overall the final prototype achieves the task of transferring files across a network. However there are many aspects that could be altered and improved on in the future. It also indicates that newer protocols using modern technologies could benefit over the already existing old solutions. As seen this prototype was able to transfer files at a greater speed compared to the tested existing solutions. Investigating further in the future possibly using the extra features spoken about above could be foundation of creating a truly modern and better protocol.
