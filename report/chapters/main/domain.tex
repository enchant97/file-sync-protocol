After testing the final prototype, it's performance and ability to be used for it's intended domain can now be discussed.

The prototype basis's itself from both types of transfer methodologies, being a combination of a streaming and block level protocol. With the ability to limit the amount of acknowledgements that are required to be sent to a minimum; by using them only for acknowledging control (request) packets and allowing many data chunks to be verified in a single packet, could potentially reduce the total time on a high latency network; allowing for a higher transfer speed.

Compared to the investigated existing protocols, the prototype is able to recover from lost and out-of-order packets, which is requirement for a file synchronisation domain as lost or incorrect data would result in a file being useless.

The prototype at it's current stage of development seems unsuited for transferring larger files, as shown in the photo test, where the amount of overhead was significant compared to existing protocols. However this prototype seems more suited to transfer smaller files such as plain-text data. In this aspect it greatly improves over the traditional protocols FTP and SMB, however rsync still seems to have the least amount of overhead. On a network with limited packet loss the prototype requires less waiting for acknowledgements during file data exchange, because of it using UDP with many of the chunked packets being acknowledged in a single request packet, which in testing has shown to increase total transfer speed drastically.

In this report only a few of the existing solutions were selected. Due to the amount of data being stored in enterprise situations, a single machine housing all of the data is impossible. Large enterprises have migrated to using distributed storage using services such as GlusterFS and Ceph. These services can be configured to create a clustered storage system with self-healing functionality using many nodes. Investigating into distributed storage systems and applying modifications to the prototype to feature live updates (for example using fsnotify) could result in a clearer need to keep developing this prototype; since low latency is vital for replication and synchronisation \parencite{broomfieldone} \parencite{aye2015platform} \parencite{weil2006ceph}.

Using this prototype in the file synchronisation domain, rsync still seems a more suitable choice as it has a feature that this prototype lacks. The ability to determine the difference of a file and only send the changes to that file, removing the need to send a whole file repeatedly on update; which greatly reduces the amount of data transferred since only the changes are sent. In a later prototype this could be added during the file handshake stage. However the protocol would have to be changed to a more block level methodology with each block being an equal size, since checksums would need to be exchanged to determine the file difference at either-end.

A feature none of the investigated protocols have; is the ability to have a long running connection with no constant ping packets to maintain the connection. This is possible as UDP is being used and each client that connects is given a unique ID that could potentially be stored in a database for intermittent file transfers without the need to constantly establish a new connection, removing the need to keep sending handshakes. A use case where "connection-less" long running file transfers could be used is synchronisation of files across a distributed file-system; where file system changes are indeterminate, unlike a backup which would happen on a schedule, and changes need to be propagated as soon as possible amongst many nodes. Removing the handshake would eliminate the setup time for when a transfer needed to be issued and also eliminate a constant ping to maintain a connection.
