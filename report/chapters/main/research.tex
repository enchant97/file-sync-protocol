\section{The Transport Layer}
Currently as of 2023 networked devices commonly use \acrfull{tcpip}. This includes the \acrfull{tcp} and the \acrfull{udp}.

\subsection*{UDP}
UDP is the oldest protocol at the transport layer. It is also the least complex, since it features only necessary parts to allow transmission of packets across networks. It is a connectionless protocol, meaning it has no handshake to establish a new connection and has no form of reliability or other error handling \parencite{udp-rfc768}.

This would make UDP on it's own is unsuitable for any reliable communication on it's own. Early use-cases were for real-time video/audio streaming and games.

This would seem to make UDP unsuitable for any reliable communication. However since protocols can be built on top of UDP to implement reliability and other features specifically targeted for the task, in recent years it has seen more use for forming the basis of reliable data transfer.

For example a modern protocol that has widespread use today is QUIC. It is built on top of UDP, providing all the necessary reliability features in the application layer. It is designed to have improved performance over TCP. Currently it is most used for serving website content over http/3 which uses QUIC \parencite{carlucci2015http}.

UDP is also used in \acrfull{vpn} protocols such as WireGuard and OpenVPN. These also have reliability built-in allowing for low latency access of remote networks.

Using UDP for reliable applications is possible since most networks have limited packet loss, due to more modern network technology being used such as switches; eliminating packet collisions.

It is now common for newer protocols to be built using UDP since it is wasteful having TCP acknowledgements for every packet since it increases the latency, due to the constant pausing for acknowledgements.

Despite UDP having no form of error correction, it does have a single checksum field in the header. Depending on the situation it can even be disabled. This checksum is usually only validates the headers are not corrupted leaving the payload to possibly have been corrupted. An under-used part of the specification is to enable the checksum to include the payload as well, meaning that the whole packet could be validated and ignored if corruption occurred.

Using the checksum field for payload validation allows for corruption to be detected directly via hardware such as the \acrfull{nic} instead of implementing a application level one. This would allow a packet to be discarded before it even reaches the app. Removing the need to check for corruption at the application layer reduces the amount of possible scenarios to handle. Most UDP applications that need to be reliable would need to handle packet loss, reordering and duplication \parencite{steinke2001tcp}.

For UDP to have the same reliability as TCP, it could be reimplemented to match TCP, however that would not improve over what already exists, as TCP could just of been used. Instead a custom solution built specifically for the task could be built. As mentioned before most internal networks now have limited packet loss, meaning that selective error checking could be implemented, allowing for a lower latency and higher throughput transfer.

\subsection*{TCP}
Building from UDP; TCP is the most commonly used protocol. Providing many features such as detection of lost packets and handling of out of order packets. All of this can be implemented and processed at the kernel level, reducing the need for a developer to implement it in their application \parencite{tcp-rfc793}.

This in-built reliability would of been important at the time most of these protocols were created, since machines at that time were limited on the amount of processing power and memory available. The reliability of TCP would have been important as many networks would have been using network configurations such as a BUS or RING to link computers together which were renowned for having packet loss due to packet collisions \parencite{steinke2001tcp}.

TCP is still the most used protocol because of the in-built features, meaning that developers of programs do not need to worry about implementing their own error checking. Because of the widespread use it also guaranties device support, meaning greater compatibility.

\subsection*{SCTP}
There are many modern protocols being developed, one of these is \acrfull{sctp}. This protocols goal is to improve on the TCP and UDP drawbacks. Offering reliable in-order data transfer while having a simpler packet structure compared to TCP, having two main sections; the header and chunks. There can be multiple chunks per packet with two different types available, payload data and control messages, this allows for smaller messages to be bundled together if they can fit inside one packet; thus reducing network overhead. \acrshort{sctp} keeps a connection open by using "heartbeat" messages, this ensures both ends of a connection knows whether they can still access each other \parencite{sctp-rfc9260} \parencite{ladha2004improving}.

SCTP seems like a suitable improvement over TCP and UDP, however it has drawbacks. Mainly it's limited adoption which is most likely due to the \acrfull{rfc} still being in a "Proposal" stage. This is a problem since it is a transmission protocol, it requires all receiving devices on the network to understand it; this would include routers and switches. This limited adoption would therefore cause issues as you would have to ensure that all devices supported it, otherwise packets may be detected as unknown and dropped by the unsupported devices; due to packets being treated as corrupted or malicious \parencite{stewart2001sctp}.

This limited supports means that it is currently unsuited for use, until more devices have support and the protocol is fully standardised and can then be supported by more network devices.

\subsection*{Conclusion}
Using \acrshort{udp} seems suitable for implementing a modern protocol; the reasons are listed below:

\begin{itemize}
    \item Custom reliability features, specific for the application can be built
    \item Offers the smallest feature set, allowing for it to be built upon
    \item Is well supported as it is part of the TCP/IP stack
\end{itemize}


\section{Data Format}
Before experimental prototypes can be made, investigating how to structure the data which will be contained in the packets payload should be investigated.

\subsection*{Telnet Strings}
Telnet Strings use plain-text encoded using \acrshort{ascii}; this would allow the protocol to work on archaic devices. However modern devices are being targeted, making this format unsuitable since it would increase the amount of network overhead required for sending ASCII messages. It would also not have a way of easily representing structured data.

\subsection*{JSON}
\acrfull{json} is the most common format seen in web based technologies, however it would be unsuitable for a protocol; since it requires all data to be encoded as ASCII so any binary data would have to be encoded for example using base64. This extra encoding step would increase both complexity and the required processing power needed to handle each packet \parencite{json-rfc8259}.

\subsection*{MessagePack}
MessagePack (or CBOR) is also another format that could be used, it is more suitable for a protocol since it is encoded directly into binary, also removing the need to encode binary data as this data can be stored directly as a bytes type. However like JSON it is also schema-less. So extra validation would have to occur when checking validity of a message \parencite{msgpack} \parencite{cbor-rfc8949}.

\subsection*{Protocol Buffers}
A newer format is protobuf. This format is designed specifically for serializing structured data using pre-defined schemas, meaning validation of a message can be checked easily, since the expected format is known from the created schemas. Having a required schema makes it more suitable for use in protocols which have a pre-defined structure. Like MessagePack it is also a "binary wire" format making the serialized result as small as possible \parencite{protobuf-3}.

\subsection*{Conclusion}
After investing possible ways of formatting data, using Protocol Buffers seem the most suitable for this project; the reasons are listed below:

\begin{itemize}
    \item Requires schema's to be pre-defined
    \item Do not require extra serialization/deserialization step when sending binary data
    \item Uses the "binary-wire" format, creating compact messages; reducing the size of the final output
    \item Reduced processor overhead required; compared to schema-less
\end{itemize}


\section{Language}
After investigating several potential programming languages, Go seems suitable for prototyping; the reasons are listed below:

\begin{itemize}
    \item Compiled language, so will be performant enough to test against the existing protocols
    \item Features garbage collection, so the prototypes can be implemented without having to consider memory management (like ownership or borrowing)
    \item Protocol Buffers have "first-class" support, since they are both written and maintained by Google
\end{itemize}

Other languages were considered such as C which most of the existing solutions are written in however it would require too much time to be taken in memory management; which would slow down prototyping. A dynamic and interpreted language Python was also considered, however it would not be performant enough for comparing against existing solutions and may have dropped more UDP packets since processing would have been slower.
