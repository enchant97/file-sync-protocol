Once built all prototypes operate the same, the single built binary contains both the client and server. The commands to run both are shown below:

Run Server:

\begin{lstlisting}
./prototype server 127.0.0.1:9000
\end{lstlisting}

Run Client:

\begin{lstlisting}
./prototype client 127.0.0.1:9000 <file path> ...

or

./prototype client 127.0.0.1:9000 <directory path>
\end{lstlisting}

The prototypes have several configurations that can be controlled; to aid testing and experimentation. These are set using environment variables.

Prototype One:

\begin{itemize}
    \item NET\_MTU, The maximum packet size that can be received
\end{itemize}

Prototype Two:

\begin{itemize}
    \item NET\_MTU, The maximum packet size that can be received
    \item CHUNKS\_PER\_BLOCK, the number of chunks that can be sent per block
\end{itemize}

Prototype Three:

\begin{itemize}
    \item NET\_MTU, The maximum packet size that can be received
    \item CHUNKS\_PER\_BLOCK, the number of chunks that can be sent per block
    \item TIMEOUT\_MS, wait time for ACK until another packet is sent
\end{itemize}
