A lot as changed since the \acrfull{tcpip} was introduced globally (on Jan 1, 1983) replacing \acrfull{ncp}. We now rely heavily on computers to aid us in both in the business sector and for personal use. A vast amount of data is transferred, spanning across large networks powering the "cloud" \parencite{tcpip-transition}.

However with the greater demand and improvements to technologies there has been little improvement to existing lower level technologies such as the core protocols. All of this data is backed up using well established (and old) protocols such as the FTP family, SMB and rsync that are all utilising the TCP protocol.

A modern solution could be very beneficial to be more efficient and fully utilise modern technologies that are available, such as multi-gigabit networks. Whilst developing a new protocol removing compatibility for archaic devices will be done; this will ensure any improvement is not hindered by supporting them.

This report will investigate improving the domain of file synchronisation at the network level by building a prototype using modern methodologies, such as building on-top of UDP and adding a custom reliability layer to reduce the amount of extra data sent over the network. The prototype will then be compared against the investigated existing solutions (FTP, SMB and rsync) to see if a improvement is seen and whether more investigation should be done in the future.
